% generated by GAPDoc2LaTeX from XML source (Frank Luebeck)
\documentclass[a4paper,11pt]{report}

\usepackage{a4wide}
\sloppy
\pagestyle{myheadings}
\usepackage{amssymb}
\usepackage[latin1]{inputenc}
\usepackage{makeidx}
\makeindex
\usepackage{color}
\definecolor{FireBrick}{rgb}{0.5812,0.0074,0.0083}
\definecolor{RoyalBlue}{rgb}{0.0236,0.0894,0.6179}
\definecolor{RoyalGreen}{rgb}{0.0236,0.6179,0.0894}
\definecolor{RoyalRed}{rgb}{0.6179,0.0236,0.0894}
\definecolor{LightBlue}{rgb}{0.8544,0.9511,1.0000}
\definecolor{Black}{rgb}{0.0,0.0,0.0}

\definecolor{linkColor}{rgb}{0.0,0.0,0.554}
\definecolor{citeColor}{rgb}{0.0,0.0,0.554}
\definecolor{fileColor}{rgb}{0.0,0.0,0.554}
\definecolor{urlColor}{rgb}{0.0,0.0,0.554}
\definecolor{promptColor}{rgb}{0.0,0.0,0.589}
\definecolor{brkpromptColor}{rgb}{0.589,0.0,0.0}
\definecolor{gapinputColor}{rgb}{0.589,0.0,0.0}
\definecolor{gapoutputColor}{rgb}{0.0,0.0,0.0}

%%  for a long time these were red and blue by default,
%%  now black, but keep variables to overwrite
\definecolor{FuncColor}{rgb}{0.0,0.0,0.0}
%% strange name because of pdflatex bug:
\definecolor{Chapter }{rgb}{0.0,0.0,0.0}
\definecolor{DarkOlive}{rgb}{0.1047,0.2412,0.0064}


\usepackage{fancyvrb}

\usepackage{mathptmx,helvet}
\usepackage[T1]{fontenc}
\usepackage{textcomp}


\usepackage[
            pdftex=true,
            bookmarks=true,        
            a4paper=true,
            pdftitle={Written with GAPDoc},
            pdfcreator={LaTeX with hyperref package / GAPDoc},
            colorlinks=true,
            backref=page,
            breaklinks=true,
            linkcolor=linkColor,
            citecolor=citeColor,
            filecolor=fileColor,
            urlcolor=urlColor,
            pdfpagemode={UseNone}, 
           ]{hyperref}

\newcommand{\maintitlesize}{\fontsize{50}{55}\selectfont}

% write page numbers to a .pnr log file for online help
\newwrite\pagenrlog
\immediate\openout\pagenrlog =\jobname.pnr
\immediate\write\pagenrlog{PAGENRS := [}
\newcommand{\logpage}[1]{\protect\write\pagenrlog{#1, \thepage,}}
%% were never documented, give conflicts with some additional packages

\newcommand{\GAP}{\textsf{GAP}}

%% nicer description environments, allows long labels
\usepackage{enumitem}
\setdescription{style=nextline}

%% depth of toc
\setcounter{tocdepth}{1}





%% command for ColorPrompt style examples
\newcommand{\gapprompt}[1]{\color{promptColor}{\bfseries #1}}
\newcommand{\gapbrkprompt}[1]{\color{brkpromptColor}{\bfseries #1}}
\newcommand{\gapinput}[1]{\color{gapinputColor}{#1}}


\begin{document}

\logpage{[ 0, 0, 0 ]}
\begin{titlepage}
\mbox{}\vfill

\begin{center}{\maintitlesize \textbf{ 4ti2gap \mbox{}}}\\
\vfill

\hypersetup{pdftitle= 4ti2gap }
\markright{\scriptsize \mbox{}\hfill  4ti2gap  \hfill\mbox{}}
{\Huge \textbf{ \textsf{GAP} wraper for 4ti2 \mbox{}}}\\
\vfill

{\Huge  0.0.2 \mbox{}}\\[1cm]
{ 06/03/2015 \mbox{}}\\[1cm]
\mbox{}\\[2cm]
{\Large \textbf{ Alfredo S{\a'a}nchez-R.-Navarro\\
    \mbox{}}}\\
{\Large \textbf{ Pedro A. Garc{\a'\i}a-S{\a'a}nchez\\
    \mbox{}}}\\
\hypersetup{pdfauthor= Alfredo S{\a'a}nchez-R.-Navarro\\
    ;  Pedro A. Garc{\a'\i}a-S{\a'a}nchez\\
    }
\end{center}\vfill

\mbox{}\\
{\mbox{}\\
\small \noindent \textbf{ Alfredo S{\a'a}nchez-R.-Navarro\\
    }  Email: \href{mailto://alfredo.sanchez@uca.es} {\texttt{alfredo.sanchez@uca.es}}\\
  Homepage: \href{www.uca.es} {\texttt{www.uca.es}}\\
  Address: \begin{minipage}[t]{8cm}\noindent
 Jerez de La Frontera \\
 \end{minipage}
}\\
{\mbox{}\\
\small \noindent \textbf{ Pedro A. Garc{\a'\i}a-S{\a'a}nchez\\
    }  Email: \href{mailto://pedro@ugr.es} {\texttt{pedro@ugr.es}}\\
  Homepage: \href{http://www.ugr.es/local/pedro} {\texttt{http://www.ugr.es/local/pedro}}\\
  Address: \begin{minipage}[t]{8cm}\noindent
 Departamento de {\a'A}lgebra,\\
 Universidad de Granada,\\
 18071 Granada, Espa{\~n}a \end{minipage}
}\\
\end{titlepage}

\newpage\setcounter{page}{2}
\newpage

\def\contentsname{Contents\logpage{[ 0, 0, 1 ]}}

\tableofcontents
\newpage

 
\chapter{\textcolor{Chapter }{ Introduction }}\logpage{[ 1, 0, 0 ]}
\hyperdef{L}{X7DFB63A97E67C0A1}{}
{
  How it was done, and why. }

 
\chapter{\textcolor{Chapter }{ Functions }}\logpage{[ 2, 0, 0 ]}
\hyperdef{L}{X86FA580F8055B274}{}
{
  Lala 
\section{\textcolor{Chapter }{ Gr{\"o}bner bases }}\logpage{[ 2, 1, 0 ]}
\hyperdef{L}{X7E4277497D877661}{}
{
  

\subsection{\textcolor{Chapter }{GroebnerBasis4ti2}}
\logpage{[ 2, 1, 1 ]}\nobreak
\hyperdef{L}{X7C36550E8112BBF2}{}
{\noindent\textcolor{FuncColor}{$\triangleright$\ \ \texttt{GroebnerBasis4ti2({\mdseries\slshape List, Order})\index{GroebnerBasis4ti2@\texttt{GroebnerBasis4ti2}}
\label{GroebnerBasis4ti2}
}\hfill{\scriptsize (function)}}\\


 \texttt{List} is a matrix of integers and \texttt{Order} is the ordering used to compute a Gr{\"o}bner basis. The ordering may be one
of the following: \mbox{\texttt{\mdseries\slshape lex}}, \mbox{\texttt{\mdseries\slshape grlex}}, \mbox{\texttt{\mdseries\slshape grevlex}} or a matrix specifying an ordering. The second argument is optional. 

 This function calls \mbox{\texttt{\mdseries\slshape groebner}} from \mbox{\texttt{\mdseries\slshape 4ti2}}. Notice that we are not using cost function. The output is a matrix
containing the differences of the exponents of the binomials of the
Gr{\"o}bner basis of the ideal that is the kernel of the mapping that maps the
variable $x_i$ to the monomial with exponent the column of the first argument. 

 
\begin{Verbatim}[commandchars=!@|,fontsize=\small,frame=single,label=Example]
  !gapprompt@gap>| !gapinput@GroebnerBasis4ti2([[3,5,7]],[[3,5,7]]);|
  [ [ -4, 1, 1 ], [ -3, -1, 2 ], [ -1, 2, -1 ] ]
  !gapprompt@gap>| !gapinput@GroebnerBasis4ti2gmp([[5, 21, 23, 26, 69]]);        |
  [ [ -13, 2, 1, 0, 0 ], [ -8, 3, -1, 0, 0 ], [ -5, -1, -1, 0, 1 ], 
    [ -5, -1, 2, 0, 0 ], [ -1, -1, 0, 1, 0 ] ]
  !gapprompt@gap>| !gapinput@GroebnerBasis4ti2([[3,5,7]],"lex");    |
  [ [ 0, 7, -5 ], [ 1, -2, 1 ], [ 1, 5, -4 ], [ 2, 3, -3 ], [ 3, 1, -2 ], 
  [ 4, -1, -1 ] ]
  !gapprompt@gap>| !gapinput@GroebnerBasis4ti2([[3,5,7]],"grevlex");|
  [ [ -1, 2, -1 ], [ 3, 1, -2 ], [ 4, -1, -1 ] ]
  !gapprompt@gap>| !gapinput@GroebnerBasis4ti2([[3,5,7]],"grlex");  |
  [ [ 0, 7, -5 ], [ 1, -2, 1 ], [ 1, 5, -4 ], [ 2, 3, -3 ], [ 3, 1, -2 ], 
  [ 4, -1, -1 ] ]
               
\end{Verbatim}
 }

 }

 
\section{\textcolor{Chapter }{ Hilbert bases }}\logpage{[ 2, 2, 0 ]}
\hyperdef{L}{X7E6DDC877C53B365}{}
{
  

\subsection{\textcolor{Chapter }{HilbertBasis4ti2}}
\logpage{[ 2, 2, 1 ]}\nobreak
\hyperdef{L}{X84B370917848E2EC}{}
{\noindent\textcolor{FuncColor}{$\triangleright$\ \ \texttt{HilbertBasis4ti2({\mdseries\slshape problem})\index{HilbertBasis4ti2@\texttt{HilbertBasis4ti2}}
\label{HilbertBasis4ti2}
}\hfill{\scriptsize (function)}}\\


 \texttt{problem} is a list to specify the input components of the "problem" (see the 4ti2
manual). 
\begin{Verbatim}[commandchars=!@|,fontsize=\small,frame=single,label=Example]
  !gapprompt@gap>| !gapinput@problem:=[ "mat", [ [ 1, -3, 9, -1, 81 ], |
  [ -9, 27, 81, -9, 1 ], [ 1, -3, 9, 1, -81 ] ], 
  "rel", [ "=", "<", ">"], "sign", [1, 0, 1, 0, -1] ];; 
  HilbertBasis4ti2(problem);
  [ [ 3, 1, 0, 0, 0 ], [ 0, -364, 0, -1095, -27 ], [ 1, 0, 0, 1, 0 ], 
    [ 1, -121, 0, -365, -9 ], [ 0, -1, 0, 3, 0 ], [ 0, -27, 0, -81, -2 ], 
    [ 0, -14, 0, -39, -1 ], [ 2, -13, 0, -40, -1 ], [ 0, 0, 1, 9, 0 ] ]
  !gapprompt@gap>| !gapinput@problem:=["mat", [[1, 1, 1, -1, -1, -1, 0, 0, 0 ], |
  [1, 1, 1, 0, 0, 0, -1, -1, -1], [0, 1, 1, -1, 0, 0, -1, 0, 0], 
  [1, 0, 1, 0, -1, 0, 0, -1,  0], [ 1, 1, 0, 0, 0, -1, 0, 0, -1], 
  [ 0, 1, 1, 0, -1, 0, 0, 0, -1], [ 1, 1, 0, 0, -1, 0, -1, 0, 0]], 
  "rel", [[0, 0 ,0 ,0, 0, 0, 0]], 
  "sign", [[0, 0, 0, 0, 0, 0, 0, 0, 0]]];; 
  HilbertBasis4ti2(problem);
  [ ]
  !gapprompt@gap>| !gapinput@problem:=["mat", [[1, -31, -1, 1], [-111, 5, 10, 25]]];;|
  HilbertBasis4ti2(problem);
  [ [ 35, 0, 136, 101 ], [ 15, 1, 36, 52 ], [ 195, 34, 0, 859 ], 
    [ 110, 19, 4, 483 ], [ 25, 4, 8, 107 ] ]
              
\end{Verbatim}
 }

 }

 
\section{\textcolor{Chapter }{ Graver bases }}\logpage{[ 2, 3, 0 ]}
\hyperdef{L}{X81572A4B7F68711A}{}
{
  

\subsection{\textcolor{Chapter }{GraverBasis4ti2}}
\logpage{[ 2, 3, 1 ]}\nobreak
\hyperdef{L}{X85A683EF847E5E50}{}
{\noindent\textcolor{FuncColor}{$\triangleright$\ \ \texttt{GraverBasis4ti2({\mdseries\slshape List})\index{GraverBasis4ti2@\texttt{GraverBasis4ti2}}
\label{GraverBasis4ti2}
}\hfill{\scriptsize (function)}}\\


 \texttt{List} is a list with two elements; the first is either \texttt{"mat"} or \texttt{"lat"}, and the second is a matrix \$\mbox{\texttt{\mdseries\slshape A}}. 

 If the first element is \texttt{"mat"}, then the output is a Graver basis of the kernel of \mbox{\texttt{\mdseries\slshape A}}. If it is \texttt{"lat"}, then the output is a Graver basis of the subgroup spanned by the rows of \mbox{\texttt{\mdseries\slshape A}}. 
\begin{Verbatim}[commandchars=!@|,fontsize=\small,frame=single,label=Example]
  !gapprompt@gap>| !gapinput@GraverBasis4ti2(["mat",[[3,5,7]]]);|
  [ [ 3, 1, -2 ], [ 7, 0, -3 ], [ 4, -1, -1 ], [ 1, -2, 1 ], [ 2, 3, -3 ], 
    [ 1, 5, -4 ], [ 0, 7, -5 ], [ 5, -3, 0 ] ]
  !gapprompt@gap>| !gapinput@GraverBasis4ti2(["lat",[ [ 1, -2, 1 ], [ -3, -1, 2 ], [ -4, 1, 1 ] ]]);|
  [ [ 1, -2, 1 ], [ 0, 7, -5 ], [ 1, 5, -4 ], [ 2, 3, -3 ], [ 3, 1, -2 ], 
    [ 4, -1, -1 ], [ 7, 0, -3 ], [ 5, -3, 0 ] ]
              
\end{Verbatim}
 }

 }

 }

 \def\indexname{Index\logpage{[ "Ind", 0, 0 ]}
\hyperdef{L}{X83A0356F839C696F}{}
}

\cleardoublepage
\phantomsection
\addcontentsline{toc}{chapter}{Index}


\printindex

\newpage
\immediate\write\pagenrlog{["End"], \arabic{page}];}
\immediate\closeout\pagenrlog
\end{document}
